%% LyX 2.1.4 created this file.  For more info, see http://www.lyx.org/.
%% Do not edit unless you really know what you are doing.
\documentclass[10pt,a4paper]{report}
\usepackage[latin1]{inputenc}
\setcounter{secnumdepth}{3}
\setcounter{tocdepth}{3}
\usepackage{amsmath}
\usepackage{amssymb}
\usepackage[unicode=true,pdfusetitle,
 bookmarks=true,bookmarksnumbered=false,bookmarksopen=false,
 breaklinks=false,pdfborder={0 0 1},backref=false,colorlinks=false]
 {hyperref}

\makeatletter

%%%%%%%%%%%%%%%%%%%%%%%%%%%%%% LyX specific LaTeX commands.
\pdfpageheight\paperheight
\pdfpagewidth\paperwidth


%%%%%%%%%%%%%%%%%%%%%%%%%%%%%% User specified LaTeX commands.
\usepackage{amsfonts}

\usepackage{float}
\usepackage{graphicx}
\usepackage{titlesec}
\usepackage{blindtext}

\usepackage{listings}
\usepackage{color}
 
\definecolor{codegreen}{rgb}{0,0.6,0}
\definecolor{codegray}{rgb}{0.5,0.5,0.5}
\definecolor{codepurple}{rgb}{0.58,0,0.82}
\definecolor{backcolour}{rgb}{0.95,0.95,0.92}
 
\lstdefinestyle{java}{
    backgroundcolor=\color{backcolour},   
    commentstyle=\color{codegreen},
    keywordstyle=\color{blue},
    numberstyle=\tiny\color{codegray},
    stringstyle=\color{codepurple},
    basicstyle=\footnotesize,
    breakatwhitespace=false,         
    breaklines=true,                 
    captionpos=b,                    
    keepspaces=true,                 
    numbers=left,                    
    numbersep=5pt,                  
    showspaces=false,                
    showstringspaces=false,
    showtabs=false,                  
    tabsize=2
}

\definecolor{gray75}{gray}{0.75}


\lstset{language = Java, style=java}


\newcommand{\hsp}{\hspace{18pt}}
\titleformat{\chapter}[hang]{\Huge\bfseries}{\thechapter\hsp\textcolor{gray75}{|}\hsp}{0pt}{\Huge\bfseries}
\author{Losio Davide Francesco, Luchetti Mauro, Mosca Paolo}
\title{Integration Test Plan Document\\ Version 1.0}

\makeatother

\begin{document}
\maketitle \tableofcontents{}


\chapter{Introduction }


\section{Revision History}
\begin{itemize}
\item January 2016 - Integration Test Plan Document: Version 1.0
\end{itemize}

\section{Purpose and Scope}

This document provides the testing plan for the integration of the
whole MTS system. Components implementation order, interfaces between
them and tests that must be executed are stated here. This document
should be used as a reference of the testing methodologies used by
the MTS team. The scope is the creation of a workflow sequence of
operation that allows the testers to work optimizing the available
time.


\section{List of Definitions and Abbreviations }
\begin{itemize}
\item \textbf{MTS}: \textbf{M}y\textbf{T}axi\textbf{S}ervice
\item \textbf{RASD:} \textbf{R}equirements \textbf{S}pecification \textbf{A}nalysis
\textbf{D}ocument 
\item \textbf{DD: D}esign \textbf{D}ocument
\end{itemize}

\section{List of Reference Documents}
\begin{itemize}
\item {\textbf{MyTaxyService RASD}} 


- November/6/2015-

\item {\textbf{MyTaxyService DD}} 


- December/4/2015-

\item The project description 
\item The documentation of any tool you plan to use for testing 
\end{itemize}

\chapter{Integration Strategy }


\section{Entry Criteria}

Before the beginning of the procedures stated in this document, it
is mandatory to develop suitable unit tests for each class. Furthermore,
this tests, have to be implemented during the class development in
order to ensure their correction and to certify that the requirements
of the class functionality are respected. The developing of the unit
tests also aims to guarantee a minimal automated testing environment,
to ensure that each revision to the code can be tested immediately.
Classes that haven't been tested cannot be included in the integration
testing, This in order to simplify error-finding procedures in addition
to trying to avoid some sort of big-bang approach. Not completed components,
either with respect to the code or the testing, will be integrated
and tested incrementally. 


\section{Elements to be Integrated}

Elements to be integrated are those individuated in DD document (look
at Reference, at point Two). For the sake of legibility they are reported
here:
\begin{itemize}
\item Queue Handler
\item Requests Handler
\item Shareable Ride Finder
\item Mapping Features Handler
\item DBMS and Log-In Handler
\item Ban Handler 
\item Passenger Communicator 
\item Driver Communicator 
\item Passenger-views 
\item Driver-views
\end{itemize}
\newpage

\begin{flushleft}
The integration testing involves all the code modules developed, moreover,
the end-user application has to be tested in the various system environments
stated in DD document at the Scope chapter, these are:
\par\end{flushleft}
\begin{itemize}
\item Unix based OS.
\item Windows systems.
\item Android.
\item iOS.
\item Windows phone.
\end{itemize}

\section{Integration Strategy}

The integration strategy will follow a mixed-approach, also known
as sandwich approach. \\\\Model integration will start with the completion
of DBMS and Log-In Handler, Mapping Features Handler and the Requests
Handler. These components can be easily tested in their integration
with the the Queue Handler, that will be initilially implemented as
a stub component and than integrated in the final realization. \\\\
Indeed, since these components require from the Queue handler only
some simple return values, this values can be retrieved by very simple
stubs class (of about one return method). \\\\This approach also
allows the functions incremental integration, in fact the Shareable
Ride Finder, which is a component bounded to an ``extra'' functionality,
will be implemented in a second time. Since it isn't an essential
part but an additional functionality, we have no need to use stubs
to fulfill his functions.\\\\ Actually, the Model part of the system
can be viewed as a ``sub-component'', in an ideal hierarchy, with
the controller and the view on top. These two parts can be mocked
up easily, to test an embryonic version of the model, which is the
core part of the entire system; this has to be developed in first
place. \\\\Here we are using a bottom-up approach instead of testing
the component of the model we have preferred a top-down approach.
This definitely justify the general sandwich approach as a proper
one. \\\\A little note has to be made on database component testing:
this will be done populating the DBMS table with some fake values,
and actually it couldn't be unit tested as a normal class. Instead,
the class interacting with him will be tested to ensure his correct
implementation.\\\\ The main feature that this strategy guarantees
is that it perfectly matches the incremental approach needed by the
developing of the system.


\section{Software Integration Sequence}

The integration will follow the path specified below:
\begin{itemize}
\item In a first time, model components will be developed and tested along
side their completion.

\begin{enumerate}
\item The first components to be developed and tested will be the DBMS \&
Log-In Handler, Mapping Features Handler and Requests Handler, as
these are the main components of the entire system. Interactions among
the related components will be simulated by the realization of simple
and quick stubs.\\\\The development procedure will be split up among
three different developing-team. This will optimize the project timeliness
since that works will going on in a parallel way. \\\\The integration
between Mapping Features Handler and the Google Maps API has to be
taken in account and carried out. Since this SOA service is already
on the market, there is no need of drivers or stubs to test this integration.
A first integration between Mapping Features Handler and DBMS \& Log-In
Handler should be done instead. \\\\The Request Handler will be also
provided of a stub component, that will play the role of a simple
Shareable Request Finder.\\\\ This procedure will be even applied
for the realization of a ban handler stub, which will also be implemented
as a driver for the Request Handler.
\item After that point one is successfully completed, Queue Handler will
be developed and tested. When it correctly meets all the Entry Criteria
stated above, it will take place of the stub used at point 1, then,
the integration will be tested again.\\\\ This approach ensures to
clearly separate integration issues coming from the Mapping Features
Handler, DBMS \& Log-In Handler and Requests Handle, or the newly
integrated component. 
\item At least, the Ban Handler will be developed and tested exploiting
the already existent stub and driver.
\item Functionalities of the whole Model ``Subsystem'' will be tested
through the using of view and controller drivers. This will give an
overview on the main issues to solve in the controllers and views
implementation.\newpage
\end{enumerate}
\item In an optic of an incremental approach, Controllers and Views components
will be still developed and tested one by one.

\begin{enumerate}
\item Controllers, either for the driver or the passenger side, will be
firstly developed an tested. This allows the developers to clearly
know what are the inputs requested by the Model to perform his functions.
They actually will be the result of the integration, and huge amplification
of the drivers used in the model testing.\\\\ At the end of this
process, when the Controllers definitely meet the entry criteria and
one at time, they will take the drivers place, and will pass thorough
the integration test.
\item View will be developed upon the Controllers integration testing completion.
As well as latter, they will be the result of the drivers integration
and amplification.\\\\ At the end of this process, when the views
definitely meet the entry criteria and one at time, they will take
take the drivers place, and will pass thorough the integration test.
\end{enumerate}
\item Reached this point, the entire system will going thorough a general
integration system test, that will cover:

\begin{itemize}
\item Whole System Testing.
\item Performance testing.
\item Ultimate features testing, exploiting the main scenarios in which
the system has to work.
\end{itemize}
\item At this time, after passing all integration and acceptance testing,
a first finite product version will be released for beta testing purpose,
among a bunch of selected test-user. This version will be spoiled
of the extra feature provided by the final one. Those are Shareable
ride function and MTS API. This will allow a more flexible timetable
for the releasing date. Indeed, if the develop process encounters
some delays, it will be possible to evaluate a first basic-release
spoiled of the ``extra'' features listed above. From time to time
the last functionalities will be added when ready.
\item Meanwhile Shareable Ride Finder component will be developed, tested
and integrated.
\item API functionalities will be implemented and tested at least with some
fake implementation of their use.\newpage
\end{itemize}
For having a better overview of the Integration Sequence procedure,
a simple scheme follows:

\begin{figure}[H] \makebox[\textwidth][c]{ \includegraphics[width=\textwidth]{\string"ITPDimages/IntegrationSequence".png} } \end{figure}


\section{Subsystem Integration Sequence}

As result of the integration sequence, the subsystems taking part
in the application will be integrated in this order:
\begin{enumerate}
\item Model core functionalities.
\item Controllers.
\item Views.
\item Extra functionalities and API.
\end{enumerate}

\chapter{Individual Steps and Test Description }


\section{Integration Test 1}

\begin{center}
\begin{tabular}{|p{0.3\textwidth}||p{0.65\textwidth}|}
\hline  \textbf{Test Case Identifier} & I1\tabularnewline
\hline  \textbf{Test Items} & DBMS \& LogIn Handler -- Mapping Features Handler\tabularnewline 
\hline  \textbf{Other Stubs} & None\tabularnewline 
\hline  \textbf{Input Specification} & Method calls from DBMS \& LogIn Handler\tabularnewline 
\hline  \textbf{Output Specification} & Check that the correct methods are called in Mapping Features Handler\tabularnewline 
\hline  \textbf{Tests applied} & The tests done must verify that the component:
\begin{itemize}
\item calls the correct method for locating a driver basing on his position when his availability state changes   
\end{itemize}
\tabularnewline 
\hline  
\end{tabular}
\par\end{center}


\section{Integration Test 2}

\begin{center}
\begin{tabular}{|p{0.3\textwidth}||p{0.65\textwidth}|}
\hline  \textbf{Test Case Identifier} & I2\tabularnewline
\hline  \textbf{Test Items} & DBMS \& LogIn Handler -- Queue Handler stub\tabularnewline 
\hline  \textbf{Other Stubs} & None\tabularnewline 
\hline  \textbf{Input Specification} & Method calls from DBMS \& LogIn Handler\tabularnewline 
\hline  \textbf{Output Specification} & Check that the correct methods are called in Queue Handler stub\tabularnewline 
\hline  \textbf{Tests applied} & The tests done must verify that the component:
\begin{itemize}
\item calls the correct method for adding a driver in the queue related to his position when his availability becomes "true"  
\item calls the correct method for deleting a driver from the queue related to his position when his availability becomes "false"  
\end{itemize}
\tabularnewline 
\hline  
\end{tabular}
\par\end{center}


\section{Integration Test 3}

\begin{center}
\begin{tabular}{|p{0.3\textwidth}||p{0.65\textwidth}|}
\hline  \textbf{Test Case Identifier} & I3\tabularnewline
\hline  \textbf{Test Items} & Request Handler -- Queue Handler stub\tabularnewline 
\hline  \textbf{Other Stubs} & Shareable Request Finder\tabularnewline 
\hline  \textbf{Input Specification} & Method calls from Request Handler\tabularnewline 
\hline  \textbf{Output Specification} & Check that the correct methods are called in Queue Handler stub\tabularnewline 
\hline  \textbf{Tests applied} & The tests done must verify that the component:
\begin{itemize}
\item calls the correct method for extracting a driver from the queue related to the ingoing request starting position
\item calls the correct method for adding a driver in the queue related to the ingoing request starting position if the user doesn't confirm or if the timeout expires
\item calls the correct method for creating a shereabled request if the ingoing one asks for it
\end{itemize}
\tabularnewline 
\hline  
\end{tabular}
\par\end{center}


\section{Integration Test 4}

\begin{center}
\begin{tabular}{|p{0.3\textwidth}||p{0.65\textwidth}|}
\hline  \textbf{Test Case Identifier} & I4\tabularnewline
\hline  \textbf{Test Items} & DBMS \& LogIn Handler -- Ban Handler Stub\tabularnewline 
\hline  \textbf{Other Stubs} & None\tabularnewline 
\hline  \textbf{Input Specification} & Method calls from Queue Handler\tabularnewline 
\hline  \textbf{Output Specification} & Check that the correct methods are called in Ban Handler \tabularnewline 
\hline  \textbf{Tests applied} & The tests done must verify that the component:
\begin{itemize}
\item calls the correct method for adding a new banned user
\item calls the correct method for checking whether a user passed as parameter is banned or not
\end{itemize}
\tabularnewline 
\hline  
\end{tabular}
\par\end{center}

\newpage


\section{Integration Test 5}

\begin{center}
\begin{tabular}{|p{0.3\textwidth}||p{0.65\textwidth}|}
\hline  \textbf{Test Case Identifier} & I5\tabularnewline
\hline  \textbf{Test Items} & Model component -- Controller\tabularnewline 
\hline  \textbf{Other Stubs} & None\tabularnewline 
\hline  \textbf{Input Specification} & Method calls from Model components\tabularnewline 
\hline  \textbf{Output Specification} & Check that the correct methods are called in Controller \tabularnewline 
\hline  \textbf{Tests applied} & The tests done must verify that:
\begin{itemize}
\item The Request Handler component calls the correct method for sending a request to a given driver 
\item The Request Handler component correctly receives an ingoing request from a passenger 
\item The Mapping Features Handler component calls the correct method for sending the map informations to a given driver 
\item The Mapping Features Handler component calls the correct method for sending the map informations to a given passenger 
\item The DBMS \& LogIn Handler component correctly receives LoginData informations from the Passenger Communicator
\item The DBMS \& LogIn Handler component correctly receives LoginData informations from the Driver Communicator
\item The DBMS \& LogIn Handler component correctly answers to passenger user sending the access permission to the related page
\item The DBMS \& LogIn Handler component correctly answers to driver user sending the access permission the related page
\item The DBMS \& LogIn Handler component refuses the access to a banned user
\end{itemize}
\tabularnewline 
\hline  
\end{tabular}
\par\end{center}

\newpage


\section{Integration Test 6}

\begin{center}
\begin{tabular}{|p{0.3\textwidth}||p{0.65\textwidth}|}
\hline  \textbf{Test Case Identifier} & I6\tabularnewline
\hline  \textbf{Test Items} & Controller subsystem -- View subsystem\tabularnewline 
\hline  \textbf{Other Stubs} & None\tabularnewline 
\hline  \textbf{Input Specification} & Method calls from Controller components\tabularnewline 
\hline  \textbf{Output Specification} & Check that the correct methods are called in View\tabularnewline 
\hline  \textbf{Tests applied} & The tests done must verify that:
\begin{itemize}
\item The Driver Communicator component calls the correct method for showing the request information to a given driver view
\item The Passenger Communicator component correctly receives an ingoing request from a passenger view
\item The Driver Communicator component calls the correct method for showing the map and a possible route to a given driver view, basing on the information provided by the model 
\item The Passenger Communicator component calls the correct method for showing the map to a given passenger view, basing on the information provided by the model 
\item The Passenger Communicator component calls the correct method for showing the user related page on his device
\item The Driver Communicator component calls the correct method for showing the driver related page on his device
\item The Passenger Communicator component calls the correct method for showing the ban informations to a banned user
\end{itemize}
\tabularnewline 
\hline  
\end{tabular}
\par\end{center}


\chapter{Tools and Test Equipment Required}

The following tools will be used in order to make tests according
to those presented during the lectures. Manual testing will be also
used when no other methods will be avaiable. 
\begin{itemize}
\item Starting by considering Mockito, it will be used either where will
be necessary to isolate dependencies (i.e during Requests Handler's
building)or to test Mapping Features Handler's interaction with external
service (Google APIs); moreover, Mockito is used to make possible
the usage of a not implemented yet class. this tool allows even us
to stub a method whenever understanding if a object method answer
reply in the foresee manners (i.e ``When an immediate request is
noticed print 'OK' `` ).
\item Arquilan might be used for set up the right interaction among the
components especially in order to understand if the database works
well.
\item JMeter can be used to have consistency and fault tollerance tested.
It provides some functionalities in order to simulate a fulfilled
form or an information overload, this can be useful close to the end,
either an entire ``Booking action'' need to be tested or whereas
the system can work well under strain or a big number of incoming
request may cause a failure.
\end{itemize}

\chapter{Program Stubs and Test Data Required }

Based on the testing strategy and test design, identify any program
stubs or special test data required for each integration step.
\end{document}
